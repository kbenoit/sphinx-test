% Generated by Sphinx.
\def\sphinxdocclass{report}
\documentclass[letterpaper,10pt,english]{sphinxmanual}
\usepackage[utf8]{inputenc}
\DeclareUnicodeCharacter{00A0}{\nobreakspace}
\usepackage[T1]{fontenc}
\usepackage{babel}
\usepackage{times}
\usepackage[Bjarne]{fncychap}
\usepackage{longtable}
\usepackage{sphinx}
\usepackage{multirow}


\title{sphinx-test Documentation}
\date{March 26, 2012}
\release{0.01}
\author{Ramnath Vaidyanathan}
\newcommand{\sphinxlogo}{}
\renewcommand{\releasename}{Release}
\makeindex

\makeatletter
\def\PYG@reset{\let\PYG@it=\relax \let\PYG@bf=\relax%
    \let\PYG@ul=\relax \let\PYG@tc=\relax%
    \let\PYG@bc=\relax \let\PYG@ff=\relax}
\def\PYG@tok#1{\csname PYG@tok@#1\endcsname}
\def\PYG@toks#1+{\ifx\relax#1\empty\else%
    \PYG@tok{#1}\expandafter\PYG@toks\fi}
\def\PYG@do#1{\PYG@bc{\PYG@tc{\PYG@ul{%
    \PYG@it{\PYG@bf{\PYG@ff{#1}}}}}}}
\def\PYG#1#2{\PYG@reset\PYG@toks#1+\relax+\PYG@do{#2}}

\def\PYG@tok@gu{\let\PYG@bf=\textbf\def\PYG@tc##1{\textcolor[rgb]{0.50,0.00,0.50}{##1}}}
\def\PYG@tok@gt{\def\PYG@tc##1{\textcolor[rgb]{0.67,0.00,0.00}{##1}}}
\def\PYG@tok@gs{\let\PYG@bf=\textbf}
\def\PYG@tok@gr{\def\PYG@tc##1{\textcolor[rgb]{0.67,0.00,0.00}{##1}}}
\def\PYG@tok@cm{\def\PYG@tc##1{\textcolor[rgb]{0.13,0.55,0.13}{##1}}}
\def\PYG@tok@vg{\def\PYG@tc##1{\textcolor[rgb]{0.00,0.41,0.55}{##1}}}
\def\PYG@tok@m{\def\PYG@tc##1{\textcolor[rgb]{0.71,0.32,0.80}{##1}}}
\def\PYG@tok@mh{\def\PYG@tc##1{\textcolor[rgb]{0.71,0.32,0.80}{##1}}}
\def\PYG@tok@go{\def\PYG@tc##1{\textcolor[rgb]{0.53,0.53,0.53}{##1}}}
\def\PYG@tok@ge{\let\PYG@it=\textit}
\def\PYG@tok@gd{\def\PYG@tc##1{\textcolor[rgb]{0.67,0.00,0.00}{##1}}}
\def\PYG@tok@il{\def\PYG@tc##1{\textcolor[rgb]{0.71,0.32,0.80}{##1}}}
\def\PYG@tok@cs{\let\PYG@bf=\textbf\def\PYG@tc##1{\textcolor[rgb]{0.55,0.00,0.55}{##1}}}
\def\PYG@tok@cp{\def\PYG@tc##1{\textcolor[rgb]{0.12,0.53,0.61}{##1}}}
\def\PYG@tok@gi{\def\PYG@tc##1{\textcolor[rgb]{0.00,0.67,0.00}{##1}}}
\def\PYG@tok@gh{\let\PYG@bf=\textbf\def\PYG@tc##1{\textcolor[rgb]{0.00,0.00,0.50}{##1}}}
\def\PYG@tok@s2{\def\PYG@tc##1{\textcolor[rgb]{0.80,0.33,0.33}{##1}}}
\def\PYG@tok@nn{\let\PYG@ul=\underline\def\PYG@tc##1{\textcolor[rgb]{0.00,0.55,0.27}{##1}}}
\def\PYG@tok@no{\def\PYG@tc##1{\textcolor[rgb]{0.00,0.41,0.55}{##1}}}
\def\PYG@tok@na{\def\PYG@tc##1{\textcolor[rgb]{0.40,0.55,0.00}{##1}}}
\def\PYG@tok@nb{\def\PYG@tc##1{\textcolor[rgb]{0.40,0.55,0.00}{##1}}}
\def\PYG@tok@nc{\let\PYG@bf=\textbf\def\PYG@tc##1{\textcolor[rgb]{0.00,0.55,0.27}{##1}}}
\def\PYG@tok@nd{\def\PYG@tc##1{\textcolor[rgb]{0.44,0.48,0.49}{##1}}}
\def\PYG@tok@ne{\let\PYG@bf=\textbf\def\PYG@tc##1{\textcolor[rgb]{0.00,0.55,0.27}{##1}}}
\def\PYG@tok@nf{\def\PYG@tc##1{\textcolor[rgb]{0.00,0.55,0.27}{##1}}}
\def\PYG@tok@si{\def\PYG@tc##1{\textcolor[rgb]{0.80,0.33,0.33}{##1}}}
\def\PYG@tok@sh{\let\PYG@it=\textit\def\PYG@tc##1{\textcolor[rgb]{0.11,0.49,0.44}{##1}}}
\def\PYG@tok@vi{\def\PYG@tc##1{\textcolor[rgb]{0.00,0.41,0.55}{##1}}}
\def\PYG@tok@nt{\let\PYG@bf=\textbf\def\PYG@tc##1{\textcolor[rgb]{0.55,0.00,0.55}{##1}}}
\def\PYG@tok@nv{\def\PYG@tc##1{\textcolor[rgb]{0.00,0.41,0.55}{##1}}}
\def\PYG@tok@s1{\def\PYG@tc##1{\textcolor[rgb]{0.80,0.33,0.33}{##1}}}
\def\PYG@tok@vc{\def\PYG@tc##1{\textcolor[rgb]{0.00,0.41,0.55}{##1}}}
\def\PYG@tok@gp{\def\PYG@tc##1{\textcolor[rgb]{0.33,0.33,0.33}{##1}}}
\def\PYG@tok@ow{\def\PYG@tc##1{\textcolor[rgb]{0.55,0.00,0.55}{##1}}}
\def\PYG@tok@mf{\def\PYG@tc##1{\textcolor[rgb]{0.71,0.32,0.80}{##1}}}
\def\PYG@tok@bp{\def\PYG@tc##1{\textcolor[rgb]{0.40,0.55,0.00}{##1}}}
\def\PYG@tok@c1{\def\PYG@tc##1{\textcolor[rgb]{0.13,0.55,0.13}{##1}}}
\def\PYG@tok@kc{\let\PYG@bf=\textbf\def\PYG@tc##1{\textcolor[rgb]{0.55,0.00,0.55}{##1}}}
\def\PYG@tok@c{\def\PYG@tc##1{\textcolor[rgb]{0.13,0.55,0.13}{##1}}}
\def\PYG@tok@sx{\def\PYG@tc##1{\textcolor[rgb]{0.80,0.42,0.13}{##1}}}
\def\PYG@tok@err{\def\PYG@tc##1{\textcolor[rgb]{0.65,0.09,0.09}{##1}}\def\PYG@bc##1{\colorbox[rgb]{0.89,0.82,0.82}{##1}}}
\def\PYG@tok@kd{\let\PYG@bf=\textbf\def\PYG@tc##1{\textcolor[rgb]{0.55,0.00,0.55}{##1}}}
\def\PYG@tok@ss{\def\PYG@tc##1{\textcolor[rgb]{0.80,0.33,0.33}{##1}}}
\def\PYG@tok@sr{\def\PYG@tc##1{\textcolor[rgb]{0.11,0.49,0.44}{##1}}}
\def\PYG@tok@mo{\def\PYG@tc##1{\textcolor[rgb]{0.71,0.32,0.80}{##1}}}
\def\PYG@tok@mi{\def\PYG@tc##1{\textcolor[rgb]{0.71,0.32,0.80}{##1}}}
\def\PYG@tok@kn{\let\PYG@bf=\textbf\def\PYG@tc##1{\textcolor[rgb]{0.55,0.00,0.55}{##1}}}
\def\PYG@tok@kr{\let\PYG@bf=\textbf\def\PYG@tc##1{\textcolor[rgb]{0.55,0.00,0.55}{##1}}}
\def\PYG@tok@s{\def\PYG@tc##1{\textcolor[rgb]{0.80,0.33,0.33}{##1}}}
\def\PYG@tok@kp{\let\PYG@bf=\textbf\def\PYG@tc##1{\textcolor[rgb]{0.55,0.00,0.55}{##1}}}
\def\PYG@tok@w{\def\PYG@tc##1{\textcolor[rgb]{0.73,0.73,0.73}{##1}}}
\def\PYG@tok@kt{\let\PYG@bf=\textbf\def\PYG@tc##1{\textcolor[rgb]{0.65,0.65,0.65}{##1}}}
\def\PYG@tok@sc{\def\PYG@tc##1{\textcolor[rgb]{0.80,0.33,0.33}{##1}}}
\def\PYG@tok@sb{\def\PYG@tc##1{\textcolor[rgb]{0.80,0.33,0.33}{##1}}}
\def\PYG@tok@k{\let\PYG@bf=\textbf\def\PYG@tc##1{\textcolor[rgb]{0.55,0.00,0.55}{##1}}}
\def\PYG@tok@se{\def\PYG@tc##1{\textcolor[rgb]{0.80,0.33,0.33}{##1}}}
\def\PYG@tok@sd{\def\PYG@tc##1{\textcolor[rgb]{0.80,0.33,0.33}{##1}}}

\def\PYGZbs{\char`\\}
\def\PYGZus{\char`\_}
\def\PYGZob{\char`\{}
\def\PYGZcb{\char`\}}
\def\PYGZca{\char`\^}
\def\PYGZsh{\char`\#}
\def\PYGZpc{\char`\%}
\def\PYGZdl{\char`\$}
\def\PYGZti{\char`\~}
% for compatibility with earlier versions
\def\PYGZat{@}
\def\PYGZlb{[}
\def\PYGZrb{]}
\makeatother

\begin{document}

\maketitle
\tableofcontents
\phantomsection\label{index::doc}


Contents:


\chapter{First Steps with Sphinx}
\label{new:first-steps-with-sphinx}\label{new::doc}\label{new:welcome-to-sphinx-test-s-documentation}
This document is meant to give a tutorial-like overview of all common tasks
while using Sphinx.

The green arrows designate ``more info'' links leading to advanced sections about
the described task.


\section{Setting up the documentation sources}
\label{new:setting-up-the-documentation-sources}
The root directory of a documentation collection is called the \emph{source
directory}.  This directory also contains the Sphinx configuration file
\emph{conf.py}, where you can configure all aspects of how Sphinx reads your
sources and builds your documentation.

Sphinx comes with a script called \emph{sphinx-quickstart} that sets up a
source directory and creates a default \emph{conf.py} with the most useful
configuration values from a few questions it asks you.  Just run

\begin{Verbatim}[commandchars=\\\{\}]
\PYG{p}{\PYGZdl{}} sphinx\PYG{o}{-}quickstart
\end{Verbatim}

and answer its questions.  (Be sure to say yes to the ``autodoc'' extension.)

\begin{Verbatim}[commandchars=\\\{\}]
library\PYG{p}{(}ggplot2\PYG{p}{)}
qplot\PYG{p}{(}wt\PYG{p}{,} mpg\PYG{p}{,} data \PYG{o}{=} mtcars\PYG{p}{)}
\end{Verbatim}

\begin{Verbatim}[commandchars=\\\{\}]
theme\PYGZus{}to\PYGZus{}header\PYGZus{}html \PYG{o}{\textless{}-} \PYG{k+kr}{function}\PYG{p}{(}theme\PYG{p}{)}\PYG{p}{\PYGZob{}}
  css\PYGZus{}file \PYG{o}{=} \PYG{k+kr}{if} \PYG{p}{(}file.exists\PYG{p}{(}theme\PYG{p}{)}\PYG{p}{)} theme \PYG{k+kr}{else} \PYG{p}{\PYGZob{}}
    system.file\PYG{p}{(}\PYG{l+s}{"}\PYG{l+s}{themes"}\PYG{p}{,} sprintf\PYG{p}{(}\PYG{l+s}{"}\PYG{l+s}{\PYGZpc{}s.css"}\PYG{p}{,} theme\PYG{p}{)}\PYG{p}{,} package \PYG{o}{=} \PYG{l+s}{"}\PYG{l+s}{knitr"}\PYG{p}{)}
  \PYG{p}{\PYGZcb{}}
  \PYG{c+c1}{\PYGZsh{} css\PYGZus{}knitr = system.file('themes', '.knitr.css', package = 'knitr')}
  css\PYGZus{}knitr \PYG{o}{\textless{}-} \PYG{l+s}{'}\PYG{l+s}{\PYGZti{}/Desktop/R\PYGZus{}Projects/knitr/inst/themes/.knitr.css'}
  stringr\PYG{p}{::}str\PYGZus{}c\PYG{p}{(}c\PYG{p}{(}
    \PYG{l+s}{'}\PYG{l+s}{\textless{}style type="text/css"\textgreater{}'}\PYG{p}{,} 
    readLines\PYG{p}{(}css\PYGZus{}knitr\PYG{p}{)}\PYG{p}{,}
    readLines\PYG{p}{(}css\PYGZus{}file\PYG{p}{)}\PYG{p}{,} 
    \PYG{l+s}{'}\PYG{l+s}{\textless{}/style\textgreater{}'}\PYG{p}{)}\PYG{p}{,} 
        collapse \PYG{o}{=} \PYG{l+s}{'}\PYG{l+s}{\PYGZbs{}n'}\PYG{p}{)}
\PYG{p}{\PYGZcb{}}
\end{Verbatim}

\begin{notice}{note}{Note:}
The default role (\code{{}`content{}`}) has no special meaning by default.  You
are free to use it for anything you like, e.g. variable names; use the
{\color{red}\bfseries{}:confval:{}`default\_role{}`} config value to set it to a known role.
\end{notice}

Another way to highlight code is to doe the following:

\begin{Verbatim}[commandchars=\\\{\}]
library\PYG{p}{(}ggplot2\PYG{p}{)}
qplot\PYG{p}{(}wt\PYG{p}{,} mpg\PYG{p}{,} data \PYG{o}{=} mtcars\PYG{p}{)}
theme\PYGZus{}to\PYGZus{}header\PYGZus{}html \PYG{o}{\textless{}-} \PYG{k+kr}{function}\PYG{p}{(}theme\PYG{p}{)}\PYG{p}{\PYGZob{}}
  css\PYGZus{}file \PYG{o}{=} \PYG{k+kr}{if} \PYG{p}{(}file.exists\PYG{p}{(}theme\PYG{p}{)}\PYG{p}{)} theme \PYG{k+kr}{else} \PYG{p}{\PYGZob{}}
    system.file\PYG{p}{(}\PYG{l+s}{"}\PYG{l+s}{themes"}\PYG{p}{,} sprintf\PYG{p}{(}\PYG{l+s}{"}\PYG{l+s}{\PYGZpc{}s.css"}\PYG{p}{,} theme\PYG{p}{)}\PYG{p}{,} package \PYG{o}{=} \PYG{l+s}{"}\PYG{l+s}{knitr"}\PYG{p}{)}
  \PYG{p}{\PYGZcb{}}
  \PYG{c+c1}{\PYGZsh{} css\PYGZus{}knitr = system.file('themes', '.knitr.css', package = 'knitr')}
  css\PYGZus{}knitr \PYG{o}{\textless{}-} \PYG{l+s}{'}\PYG{l+s}{\PYGZti{}/Desktop/R\PYGZus{}Projects/knitr/inst/themes/.knitr.css'}
  stringr\PYG{p}{::}str\PYGZus{}c\PYG{p}{(}c\PYG{p}{(}
    \PYG{l+s}{'}\PYG{l+s}{\textless{}style type="text/css"\textgreater{}'}\PYG{p}{,}
    readLines\PYG{p}{(}css\PYGZus{}knitr\PYG{p}{)}\PYG{p}{,}
    readLines\PYG{p}{(}css\PYGZus{}file\PYG{p}{)}\PYG{p}{,}
    \PYG{l+s}{'}\PYG{l+s}{\textless{}/style\textgreater{}'}\PYG{p}{)}\PYG{p}{,}
        collapse \PYG{o}{=} \PYG{l+s}{'}\PYG{l+s}{\PYGZbs{}n'}\PYG{p}{)}
\PYG{p}{\PYGZcb{}}
\end{Verbatim}


\chapter{Indices and tables}
\label{index:indices-and-tables}\begin{itemize}
\item {} 
\emph{genindex}

\item {} 
\emph{modindex}

\item {} 
\emph{search}

\end{itemize}



\renewcommand{\indexname}{Index}
\printindex
\end{document}
